\section{Overview of \OOMMF}\label{sec:overview}
The goal of the
\htmladdnormallinkfoot{\OOMMF}{http://math.nist.gov/oommf/} (Object
Oriented MicroMagnetic Framework) project in the
\htmladdnormallink{Information Technology
Laboratory}{http://www.itl.nist.gov/} (ITL) at the
\htmladdnormallink{National Institute of Standards and
Technology}{http://www.nist.gov/} (NIST)
is to develop a portable, extensible public domain micromagnetic
program and associated tools.  This code forms a completely
functional micromagnetics package, with the additional capability to
be extended by other programmers so that people developing
new code can build on the \OOMMF\ foundation.
The main
contributors to \OOMMF\ are
\ifnotpdf{\htmladdnormallink{Mike Donahue}{http://math.nist.gov/\~{}MDonahue}}
\pdfonly{\htmladdnormallink{Mike Donahue}{http://math.nist.gov/\%7EMDonahue}}
and
\ifnotpdf{\htmladdnormallink{Don Porter}{http://math.nist.gov/\~{}DPorter}.}
\pdfonly{\htmladdnormallink{Don Porter}{http://math.nist.gov/\%7EDPorter}.}

\OOMMF\ is written in C++, a widely-available, object-oriented language
that can produce programs with good performance as well as extensibility.
For portable user interfaces, we make use of \Tcl/\Tk\ so that \OOMMF\ 
operates across a wide range of \Unix, \Windows, and \MacOSX\ platforms.

The code may be modified at three distinct levels.  At the top level,
individual programs interact via well-defined protocols across network
sockets\index{network~socket}.  One may connect these modules together
in various ways from the user interface, and new modules speaking the
same protocol can be transparently added.  The second level of
modification is at the \Tcl/\Tk\ script level.  Some modules allow
\Tcl/\Tk\ scripts to be imported and executed at run time, and the top
level scripts are relatively easy to modify or replace.  At the lowest
level, the C++ source is provided and can be modified, although at
present the documentation for this is incomplete (cf.\ the ``OOMMF
Programming Manual'').

%The first portion of OOMMF released was a magnetization file display
%program called
%\htmladdnormallink{\app{mmDisp}}{http://math.nist.gov/oommf/mmdisp/mmdisp.html}\index{application!mmDisp}.
%A \htmladdnormallinkfoot{working
%release}{http://math.nist.gov/oommf/software.html} of the complete OOMMF
%project was first released in October, 1998.  It included a problem
%editor, a 2D micromagnetic solver\index{simulation~2D}, and several
%display widgets, including an updated version of \app{mmDisp}.  The
%solver can be controlled by an {\hyperrefhtml{interactive
%interface}{interactive interface (Sec.~}{)}{sec:mmsolve2d}}, or through
%a sophisticated {\hyperrefhtml{batch control system}{batch control
%system (Sec.~}{)}{sec:obs}}.  This solver was originally based on a
%micromagnetic code that
%\ifnotpdf{\htmladdnormallink{Mike Donahue}{http://math.nist.gov/\~{}MDonahue}}
%\pdfonly{\htmladdnormallink{Mike Donahue}{http://math.nist.gov/\%7EMDonahue}}
%and
%\htmladdnormallink{Bob McMichael}{mailto:rmcmichael@nist.gov}
%had previously developed.  It utilizes a Landau-Lifshitz
%ODE\index{ODE!Landau-Lifshitz} solver to relax 3D spins on a 2D
%mesh\index{grid} of square cells, using FFT's\index{FFT} to compute the
%self-magnetostatic (demag) field\index{field!demag}.  Anisotropy,
%applied field\index{field!applied}, and initial
%magnetization\index{magnetization!initial} can be varied pointwise, and
%arbitrarily shaped elements\index{boundary} can be modeled.  

The current development version, \OOMMF\ 1.2, includes Oxs, the
\OOMMF\ eXtensible Solver.  Oxs offers users of \OOMMF\ the ability
to extend Oxs with their own modules.  
%The details of programming
%an Oxs extension module are found in the
%\htmladdnormallinkfoot{OOMMF Programming Manual
%}{http://math.nist.gov/oommf/doc/}.  
The extensible nature of the Oxs
solver means that its capabilities may be varied as necessary for the
problem to be solved.  Oxs modules distributed as part of \OOMMF\
support full 3D\index{simulation~3D} simulations suitable for modeling
layered materials.

{\samepage
If you want to receive e-mail\index{e-mail}
notification\index{announcements} of updates to this project, register
your e-mail address with the ``\mumag'' mailing list:
\begin{center}
\ifnotpdf{\htmladdnormallink{http://www.ctcms.nist.gov/\~{}rdm/email-list.html}{http://www.ctcms.nist.gov/\~{}rdm/email-list.html}.}
\pdfonly{\htmladdnormallink{http://www.ctcms.nist.gov/\~{}rdm/email-list.html}{http://www.ctcms.nist.gov/\%7Erdm/email-list.html}.}
\end{center}
} % end \samepage

The \OOMMF\ developers are always interested in your comments about
\OOMMF.  See the \hyperrefhtml{Credits}{Credits (Sec.~}{)}{sec:credits}
for instructions on how to contact them, and for information on
referencing \OOMMF.
